\documentclass{article}
\usepackage{enumerate}
\usepackage{algorithm}
\usepackage{algorithmic} % pesudo code
% automatical line breaking for long string
\usepackage{seqsplit}

\usepackage{geometry}
 \geometry{
 a4paper,
 total={210mm,297mm},
 left=20mm,
 right=20mm,
 top=20mm,
 bottom=20mm,
 }

\usepackage{hyperref}

\usepackage{listings}

\lstdefinestyle{customc}{
  belowcaptionskip=1\baselineskip,
  breaklines=true,
  frame=L,
  xleftmargin=\parindent,
  language=C,
  showstringspaces=false,
  basicstyle=\footnotesize\ttfamily,
  keywordstyle=\bfseries\color{green!40!black},
  commentstyle=\itshape\color{purple!40!black},
  identifierstyle=\color{blue},
  stringstyle=\color{orange},
}

\lstdefinestyle{customasm}{
  belowcaptionskip=1\baselineskip,
  frame=L,
  xleftmargin=\parindent,
  language=[x86masm]Assembler,
  basicstyle=\footnotesize\ttfamily,
  commentstyle=\itshape\color{purple!40!black},
}

\lstset{escapechar=@,style=customc}

\usepackage{tikz}
\usetikzlibrary{positioning,shapes,arrows, calc, fit, trees}
\tikzset{
    %Define standard arrow tip
    >=stealth',
    %Define style for boxes
    punkt/.style={
           rectangle,
           rounded corners,
           draw=black, very thick,
           text width=6.5em,
           minimum height=2em,
           text centered},
    % Define arrow style
    pil/.style={
           ->,
           thick,
           shorten <=2pt,
           shorten >=2pt,}
}

\usepackage{amsmath,amsfonts,amsthm, amssymb} % Math packages
\usepackage{graphicx}
\usepackage{caption}
\usepackage{subcaption}

\usepackage{dcolumn}
\usepackage{booktabs}
% for x\y style legend
\usepackage{makecell}

\newcolumntype{M}[1]{D{.}{.}{1.#1}}
\newcommand{\norm}[1]{\left\lVert #1 \right\rVert}
\newtheoremstyle{quest}{\topsep}{\topsep}{}{}{\bfseries}{}{ }{\thmname{#1}\thmnote{ #3}.}
\theoremstyle{quest}
\newtheorem*{question}{Question}

\newcommand{\argmax}{\operatornamewithlimits{argmax}}

%\DeclareFontFamily{OT1}{pzc}{}
%\DeclareFontShape{OT1}{pzc}{m}{it}%
%              {<-> s * [0.900] pzcmi7t}{}
%\DeclareMathAlphabet{\mathpzc}{OT1}{pzc}%
%                                 {m}{it}
%\DeclareMathAlphabet{\mathcalligra}{T1}{calligra}{m}{n}
% allow insert tab
%\newcommand{\tab}[1]{\hspace{.2\textwidth}\rlap{#1}}
% align without column
% http://tex.stackexchange.com/questions/98939
\newcommand*\leftright[2]{%
  \leavevmode
  \rlap{#1}%
  \hspace{0.5\linewidth}%
  #2}
\begin{document}
\begin{enumerate}
\item
	\begin{enumerate}[a.]
	\item
	Naive Bayes model.
	\item
		\includegraphics[scale=0.3]{bayesNet-1.png}
		
		I add two edges, one between person and doors, another one between buyprice and maintprince. Because I think they are related given class, e.g. if a car can hold more person, it should have more doors to be convenient; a low maintprice can to some extent reconcile a high buyprice.
	\item
	    \includegraphics[scale=0.34]{bayesNet-3.png}  (1 is maintprice)
	    
	    Changed parameters: maxNrOfParents = 5, randomOrder = True.
	    
	    This Bayes Net do not have $(buyprice \perp maintprice| class), (persons \perp safety|class), (lugsize \perp safety|class)$
	\end{enumerate}
\item
	\begin{enumerate}[a.]
	\item
	\begin{tabular}{ccccc}
	\hline
	classifier & BN(a)&BN(b) &BN(c) & ZeroR\\ \hline
	accuracy & 85.3\% & 90.9\% & 94.1\% & 70\% \\ \hline
	\end{tabular}
	\item
	$\alpha = 0$
	\item
	Because the parameter of this network was manually tuned, and since the note can have more parent, the search is more complete.
	\end{enumerate}
\item
	\begin{enumerate}[a.]
	\item
	$P(B|A=1, C=0) = P(B|A=1)$\\
	$P(B)^{t\rightarrow \infty} = 0$
	\begin{proof}
	\begin{align*}
	P(B|A=1,C=0) &= \cfrac{P(B, A=1, C=0)}{\sum\limits_B P(B, A=1, C= 0)} \\
	                      &= \cfrac{P(A=1)P(B|A=1)P(C=0|B)}{P(A=1)\sum\limits_B P(B|A=1)P(C=0|B)} \\
	                      &= \cfrac{0.5\cdot P(C=0|B)}{0.5\cdot P(C=0|B=0) + 0.5\cdot P(C=0|B=1)} \\
	                      &= P(C=0|B) 	
	\end{align*}
	
	For $B=1$, $P(B=1|A=1, C=0) = P(C=0|B=1) = 0$.
	
	Thus $B^1 =0$, then we have $C^1=0$. No matter what $A^1$ is, we still get $B^2=0, C^2=0$ (Because $P(A=0)=P(A=1)=P(B=0|A=1)=P(B=0|A=0)=P(B=1|A=0)=P(B=1|A=1)=0.5$). By induction $P(B)^{t\rightarrow \infty} = 0$
	\qedhere
	\end{proof}
	\item
	Since $P(B=0, C=1) = P(B=1,C=0) = 0$, this gibbs chain is not regular. To make it mix, we make let $P(C=1|B=1)=P(C=0|B=0)= 1- \epsilon$.
	
	Test with $\epsilon = 0.01$, number of iteration = 100000. The probabilty of A, B, C:
	
	\begin{tabular}{ccc}
	\hline
	& =0 & =1\\ \hline
	P(A) & 0.50062  & 0.49938 \\
 P(B)& 0.50883  &0.49117 \\
  P(C)&0.50909  &0.49091 \\ \hline
  
	\end{tabular}
	\lstinputlisting[caption=gibbsSampling.py, style=customc, language=Python]{gibbsSampling.py}
	\end{enumerate}
\end{enumerate}
\end{document}